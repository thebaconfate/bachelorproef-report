\documentclass[a4paper]{article}

% Load the VUB package.
% This has many options, please read the documentation at
% https://gitlab.com/rubdos/texlive-vub
\usepackage{vub}
\usepackage[english]{babel}

% Some highly suggested packages, please read their manuals.

\usepackage{float}
\usepackage{cleveref}
%\usepackage[natbib,style=apa]{biblatex}
%\addbibresource{bibliography.bib}
\setlength\parskip{\baselineskip}

\title{Python on the edge}
%\pretitle{\flushleft{Graduation thesis submitted in partial fulfilment of the requirements for the degree of bachelor in de Wetenschappen: Computerwetenschappen}}
%\subtitle{my thesis subtitle}
\author{Gérard Lichtert}
\date{\today}
\promotors{Promotors: prof.\ dr.\ Joeri de Koster \and prof.\ dr.\ Wolfgang de Meuter. \and Supervisor: Mathijs Saey}
\faculty{sciences and bioengineering sciences}

\begin{document}
\maketitle
\tableofcontents
\newpage
\raggedright{}


\section{Introduction}
The Soft lab is involved in a project to optimize energy consumption in distributed programs. To give you more context. We are talking about processing data that is generated by thousands of sensors or 'data generators'. The data is generated and then sent over the network to a server, where the data is processed. This means that there is a lot of network traffic.

By applying the edge-computing principle we could pre-process part of the generated data on the edge devices themselves, before sending it over the network. This means that not all the generated data is sent to the server, but only the data that remains or is calculated after the pre-processing. Consequently, this would reduce the network traffic and the energy consumption, or at least that is our hypothesis.
\subsection{Context}
First we need to get acquainted with the system and its environment that we will be creating a tool for. The system is an actor-based distributed system that runs actors on different machines. For the time being it runs actors on the sensors and server. The actors on the sensor are responsible for sending data to the server actors, which in turn are responsible for processing and storing the data.

Furthermore, our language of implementation will be Python. This because the company that is involved in the project already has most of the infrastructure in Python and thus tasked us with creating the tool in Python as well.

\subsection{Goals}
The goal of this bachelor thesis is to create a tool that allows us to statically deploy actors at pre-configured locations. We want to try the configuration of moving part of the processing to the sensors but would also like to try other configurations such as having some preprocessing done on some sensors, and some preprocessing done on the server. We want to be able to easily change the configuration and see the effects of the changes on the network traffic and energy consumption. Note that monitoring the network traffic and energy consumption is outside the scope of this thesis. We will only be focusing on the static deployment of the actors.
\section{Context}

\section{Conclusion}

%\printbibliography%
\end{document}
