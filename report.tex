\documentclass[a4paper]{article}

% Load the VUB package.
% This has many options, please read the documentation at
% https://gitlab.com/rubdos/texlive-vub
\usepackage{vub}
\usepackage[english]{babel}

% Some highly suggested packages, please read their manuals.

\usepackage{float}
\usepackage{cleveref}
%\usepackage[natbib,style=apa]{biblatex}
%\addbibresource{bibliography.bib}
\setlength\parskip{\baselineskip}

\title{Python on the edge}
%\pretitle{\flushleft{Graduation thesis submitted in partial fulfilment of the requirements for the degree of bachelor in de Wetenschappen: Computerwetenschappen}}
%\subtitle{my thesis subtitle}
\author{Gérard Lichtert}
\date{\today}
\promotors{Promotors: prof.\ dr.\ Joeri de Koster \and prof.\ dr.\ Wolfgang de Meuter. \and Begeleider: Mathijs Saey}
\faculty{sciences and bioengineering sciences}

\begin{document}
\maketitle
\tableofcontents
\newpage
\raggedright{}


\section{Introduction}
The Soft lab is involved in a project to optimize energy consumption in distributed programs. To give you more context. We are talking about processing data that is generated by thousands of sensors or 'data generators'. The data is generated and then sent over the network to a server, where the data is processed. This means that there is a lot of network traffic.

By applying the edge-computing principle we could pre-process part of the generated data on the edge devices themselves, before sending it over the network. This means that not all the generated data is sent to the server, but only the data that remains or is calculated after the pre-processing. Consequently, this would reduce the network traffic and the energy consumption of the network.
\subsection{Context}
First we need to get acquainted with the system and its environment that we will be creating a tool for. The system is an actor-based system that runs actors on different machines. For the time being it runs actors on the sensors and server. The actors on the sensor are responsible for sending data to the server actors, which in turn are responsible for processing and storing the data.

Furthermore, our language of implementation will be Python. This because the company that is involved in the project already has most of the infrastructure in Python and thus tasked us with creating the tool in Python as well.

\subsection{Goals}
\section{Context}

\section{Conclusion}

%\printbibliography%
\end{document}
